\documentclass[12pt]{article}
% links
% https://oeis.org/wiki/List_of_LaTeX_mathematical_symbols

% Page Setup
\usepackage{geometry}
\geometry{a4paper, margin=1in}
\usepackage{titlesec}
\titleformat{\section}{\normalfont\large\bfseries}{\thesection}{1em}{}
\titleformat{\subsection}{\normalfont\normalsize\bfseries}{\thesubsection}{1em}{}
\usepackage{parskip} % remove first line indentation for paragraphs
\usepackage{amsmath}
\usepackage[normalem]{ulem} % \sout strikethrough
\newcommand{\smath}[1]{\hbox{\sout{$#1$}}} %strikethrough for math expression
\usepackage[noend]{algpseudocode}
\usepackage{listings}
\usepackage{graphicx}

% Title and Author Information
\title{COMP90025 Assignment 2 - Gravity simulation using MPI}
\author{Di Liu (1399095)}
\date{\today}

\begin{document}

\maketitle

\section{Introduction}
% 400 words

% define the problem as above in your own words
Gravity simulation in a parallel setting
for distributed computing using MPI and openMP is balance between accuracy and
performance. While the goal of the simulation is to mimic the gravity follow newtown's
newton's gravitational law, the
implementation does not strickly follow the equation. Rather, the simulation
is a simplification of the physics of gravity, where the gravitational constant is
set to be 1 and all the weight of masss are set to be 1. Therefore, the implementation
is a simulation of the computational process for the gravity simulation.
Also to ensure the simulation stops at some point, the force between two points
is set to have a max of 1 unit. This is to ensure that the force between two points
will not be too large when 2 points are close to each other, resulting in hugly accerlated
particles outwards and increase the distance variance.

% and discuss the parallel technique that you have implemented.
For the purpose of simulation, the number of particles $N$ can grow to be very large while
the number of processes $k$ and dimesion $D$ can be small.
Hence in the implementation, the communication between processes avoids points
synchronization which can be costly. Furthermore, to make the communication more
effecient, more advanced techniques such as =Alltoallv=, =Allgather= are used.
To further improve the communication, non-blocking communication and merging
some communication can be used. However, the implementation does not use these
techniques due to time constriat. To improve the performance, =omp parallel for=
is used to parallelize the computation when applicable or when a for loop is large enough.
This applies to most the for loops loop through the particles ($n$) but does not apply
to demision $D$ loops.

Overall the implementation bias towords performance over accuracy.
Specifically, the simulation treat a cluster as a whole which does not
synchronize the points between processes even when the points are close enough.
This will lead to huge inaccuracy in the simulation
when generated clusters are closed to each other. Such choice will lead to
simplier communication logic and eaiser implementation hence was chosen.

% Cite any relevant literature that you have made use of, either as a basis for your code or for comparison.
% TODO kmeans pp
The kmeans++ algorithm is used for the more efficient clustering. No other literature
is used.


\textbf{Description:} Parallel pseudo-code for gravitational simulation \\
\textbf{Input:} number of particles $N$, number of processes $k$\\
\textbf{Output:} The graph distance $\delta$ from $s$.

\begin{algorithmic}[1]
\Procedure{kmeans++$_{EREW}^{\diamond}$}{$V, r(\cdot), s$}
  \State $\delta(\cdot) \gets \infty, \delta(s) \gets 0$
  \For{$i \in 1..RANK$}
  \EndFor
\EndProcedure
\end{algorithmic}

\section{Methodology}
% 500 words
% discuss the experiments that you will use to measure the performance
% of your program, with mathematical definitions of the performance measures
% and/or explanations using diagrams, etc.

To measure the performance of the program, we will mainly observe the CPU time
of different tasks running on different environments and inputs.

The speedup function $speedup(n) = \frac{T}{t(n)}$,
where $T$ is the CPU time taken when the number of MPI nodes $n=1$,
and $t(n)$ is the CPU time when the number of nodes is $n$.

As it is not the consider of this project, the experiments will only focus on
the speedup of the MPI part of the program. The speedup of the OpenMP part
is not measured and will be set to 4 for each node.

The controllable variables that affect the program are
the number of particles $N$, the number of clusters $k$,
the dimension $D$, the number of components $c$.

The shape of the components is also a variable that will
affect the number of iterations in kmeans and simulation.
However due to it's complexity it will not be investigated
into depth.

in this project due to limited time.
The number of iterations in kmeans clustering $I_1$
and the number of iterations in kmeans clustering $I_2$
is also a variable but is not controllable

% The total work $w(n)$ is also measured using CPU time
% and will be used to compare the work done by the program.
% To avoid inconsistency between each run, each case will be run
% 5 times and the median wall time and median CPU time will be taken.
% A Python script is developed to run this batch and generate appropriate figures.

Specifically, the following questions will be answered:

\begin{itemize}
    \item How does the number of particles $N$ affect the performance?
\\The number of particles $N$ is set to be 1000, 10000, 100000, 1000000, 10000000.
    \item How does the number of clusters $k$ affect the performance?
\\The number of clusters $k$ is set to be 1, 2, 4, 8, 16, 32, 64, 128, 256, 512.
    \item How does the dimension $D$ affect the performance?
\\The dimension $D$ is set to be 1, 2, 3, 4, 5, 6, 7, 8, 9, 10.
    \item How does the number of iterations $I$ affected by all the above parameters?
\end{itemize}

% identify the variables and parameters that you will use in your experiments
% and explain how you will vary them.

% describe the experimental setup, including the hardware and software used,
% and the data sets used.

\subsection{Experiments}
% Subsection content
% describe the results of your experiments, including any tables or graphs
% that you have used to present the results.

% discuss the results of your experiments, including any tables or graphs


\section{Conclusion}
% Your conclusion goes here

% references
% https://www.overleaf.com/learn/latex/Bibliography_management_with_bibtex
%\bibliographystyle{plain}
%\bibliography{references}

\end{document}

